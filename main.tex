\documentclass[UTF8,a4paper]{ctexart}

\usepackage{amsmath,amssymb,mathtools,bm}
\usepackage{siunitx}
\usepackage{physics}
\usepackage{hyperref}
\usepackage{geometry}
\geometry{margin=2.2cm}
\hypersetup{colorlinks=true,linkcolor=blue,urlcolor=blue,citecolor=blue}

% ---------- 常用宏 ----------
\newcommand{\kb}{k_{\mathrm B}}
\newcommand{\Teff}{T_{\mathrm{eff}}}
\newcommand{\Lsun}{L_\odot}
\newcommand{\Msun}{M_\odot}
\newcommand{\Rsun}{R_\odot}
\newcommand{\F}{\mathcal{F}}
\newcommand{\J}{\mathcal{J}}
\newcommand{\Hh}{\mathcal{H}}
\newcommand{\Kk}{\mathcal{K}}

\title{恒星大气物理:核心概念与推导(复习笔记)}
\author{}
\date{\today}

\begin{document}
\maketitle
\tableofcontents

\section{光度(Luminosity)}
\subsection*{定义与物理意义}
光度 \(L\) 是天体向所有方向辐射(或输出)能量的总功率:
\begin{equation}
L \equiv \frac{\dd E}{\dd t}\,.
\end{equation}
它是恒星的“总能量预算”指标:与内部能量产生(核反应)和外层能量输运(辐射/对流)共同决定恒星在演化轨道上的位置。

\subsection*{与通量、半径的关系}
若在距离 \(r\) 处测得辐射通量 \(F(r)\)(单位面积单位时间的能流),在各向同性近似下:
\begin{equation}
L = 4\pi r^2 F(r)\,.
\end{equation}
在恒星表面 \(r=R\):
\begin{equation}
L = 4\pi R^2 F_{\rm surf}\,.
\end{equation}

\subsection*{天文学意义}
\begin{itemize}
  \item \textbf{距离标尺}:与视亮度结合可定义绝对星等与距离模数。
  \item \textbf{结构约束}:主序星满足近似 \(L\!\sim\! M^\alpha\)(\(\alpha\approx 3\text{--}4\)),反映内部能量产生与不透明度的综合效应。
  \item \textbf{大气建模边界条件}:大气模型最终要匹配给定 \(L\)(或 \(\Teff\)),保证能量守恒。
\end{itemize}

\section{有效温度(Effective Temperature)}
\subsection*{定义}
有效温度 \(\Teff\) 由“同光度同半径的黑体”定义:
\begin{equation}
L \equiv 4\pi R^2 \sigma \Teff^4\,,
\end{equation}
因此
\begin{equation}
\Teff = \left(\frac{L}{4\pi R^2\sigma}\right)^{1/4}.
\end{equation}

\subsection*{物理意义}
\(\Teff\) 不是大气任一点的真实气体温度,而是把观测到的总辐射能流(频率积分后的通量)映射为黑体温度的“等效参数”。谱能量分布形状(\(\lambda_{\rm peak}\))、电离/激发平衡(Saha/玻尔兹曼)都强烈依赖 \(\Teff\)。

\subsection*{与颜色温度、亮温的区别}
\begin{itemize}
  \item \textbf{颜色温度}:用谱形(例如两段波段比色)拟合的温度,受谱线/消光影响。
  \item \textbf{亮温}:用某一频率处的比强度等效为黑体得到的温度,常用于射电。
  \item \textbf{有效温度}:由频率积分后的总通量定义,最接近能量守恒量。
\end{itemize}

\section{赫罗图(H--R 图)}
\subsection*{坐标与基本结构}
赫罗图以 \(\log \Teff\)(或颜色指数)为横轴、\(\log L\)(或绝对星等)为纵轴,揭示恒星族群的分布规律:主序、巨星支、白矮、水平支等。

\subsection*{与大气物理的连接}
从观测到的光谱/颜色推回 \(\Teff\)、表面重力 \(g\) 与金属丰度 \([\mathrm{Fe/H}]\),再映射到 \(L\) 与演化状态。大气中\textbf{连续谱与线谱形成}决定了温度刻度与重力敏感线(如压强展宽)如何被测量。

\subsection*{天文学意义}
\begin{itemize}
  \item \textbf{演化诊断}:同一星团的转折点给出年龄;巨星支形态与金属丰度相关。
  \item \textbf{恒星参数反演}:通过 \(L\) 与 \(\Teff\) 得到 \(R\),结合 \(g=GM/R^2\) 得到质量尺度。
\end{itemize}

\section{热动平衡状态(Thermodynamic Equilibrium, TE)}
\subsection*{定义}
严格热动平衡要求:物质与辐射场处处具有同一温度 \(T\),各微观过程满足\textbf{详细平衡}(forward rate = reverse rate),辐射场为各向同性黑体:
\begin{equation}
I_\nu = B_\nu(T),\qquad J_\nu=B_\nu(T).
\end{equation}
此时粒子速度分布为 Maxwell 分布,能级布居为玻尔兹曼分布,电离由 Saha 公式给出。

\subsection*{为何恒星大气通常不满足 TE}
恒星大气是\textbf{开放系统}:能量向外逃逸,辐射场随深度变化且非各向同性;外层密度低,碰撞不足以强制“物质与辐射完全耦合”,导致非局部热动平衡(NLTE)效应。

\subsection*{天文学意义}
TE 是推导 Saha/玻尔兹曼/普朗克与灰大气的基准极限;理解“偏离 TE 的方向与原因”是 NLTE 谱线形成的核心。

\section{Saha 公式(电离平衡)}
\subsection*{要解决的问题}
给定温度 \(T\) 与电子密度 \(n_e\)(或压强),求元素在相邻电离态 \(i\leftrightarrow i{+}1\) 之间的数密度比:
\(\,n_{i+1}/n_i\)。

\subsection*{结果(常用形式)}
设电离能为 \(\chi_i\)(从 \(i\) 到 \(i{+}1\)),配分函数为 \(U_i(T)\)。Saha 方程:
\begin{equation}
\frac{n_{i+1}n_e}{n_i}
=
\frac{2U_{i+1}(T)}{U_i(T)}
\left(\frac{2\pi m_e \kb T}{h^2}\right)^{3/2}
\exp\!\left(-\frac{\chi_i}{\kb T}\right).
\label{eq:saha}
\end{equation}

\subsection*{推导要点(由化学势/统计力学得到)}
在 TE 中,相邻反应
\[
\mathrm{X}^{i} \rightleftharpoons \mathrm{X}^{i+1} + e^-
\]
满足化学势平衡
\begin{equation}
\mu_i = \mu_{i+1} + \mu_e.
\end{equation}
对理想气体组分,\(\mu\) 可写为
\begin{equation}
\mu = -\kb T \ln\!\left[\frac{U(T)}{n}\left(\frac{2\pi m\kb T}{h^2}\right)^{3/2}\right],
\end{equation}
其中 \(U\) 代表内部分配函数(含简并度与激发态贡献)。把 \(\mu_i,\mu_{i+1},\mu_e\) 代入并整理,同时把电离能 \(\chi_i\) 作为内能差写入指数项,即得式 \eqref{eq:saha}。

\subsection*{物理解释}
\begin{itemize}
  \item \textbf{指数项} \(\exp(-\chi_i/\kb T)\):温度越高越易电离。
  \item \textbf{相空间因子} \(\left(2\pi m_e \kb T/h^2\right)^{3/2}\):自由电子平动状态数随 \(T^{3/2}\) 增加。
  \item \textbf{电子密度}:\(n_e\) 越大,复合越强,电离度越低(对固定总粒子数)。
\end{itemize}

\subsection*{天文学意义}
电离平衡决定谱线来自哪一电离态(例如 Fe\,I vs Fe\,II),从而使得谱线强度对 \(T\)、\(n_e\)、\(g\) 具有诊断能力(常用“电离平衡”定标表面重力)。

\section{玻尔兹曼分布(激发平衡)}
\subsection*{结果}
同一电离态 \(i\) 内,不同激发能级 \(j\) 的布居满足:
\begin{equation}
\frac{n_{i,j}}{n_i} = \frac{g_{i,j}}{U_i(T)}\exp\!\left(-\frac{E_{i,j}}{\kb T}\right),
\label{eq:boltzmann}
\end{equation}
其中 \(g_{i,j}\) 为简并度,\(E_{i,j}\) 为相对基态的激发能。

\subsection*{推导要点}
在 TE 中微观态概率 \(p_s \propto \exp(-E_s/\kb T)\)。将能级 \(j\) 的简并度 \(g_j\) 计入,得到 \(n_j \propto g_j \exp(-E_j/\kb T)\),归一化常数即配分函数 \(U(T)=\sum_j g_j \exp(-E_j/\kb T)\)。

\subsection*{物理与天文学意义}
玻尔兹曼分布说明:高激发线(大 \(E_{i,j}\))对温度极敏感,因此高激发谱线常用于 \(\Teff\) 或温度结构的约束;同时它与 Saha 联合决定“总的可吸收粒子数”。

\section{普朗克分布函数(黑体辐射)}
\subsection*{结果}
黑体在温度 \(T\) 的谱辐射亮度(源函数极限):
\begin{align}
B_\nu(T) &= \frac{2h\nu^3}{c^2}\frac{1}{\exp(h\nu/\kb T)-1},\\
B_\lambda(T) &= \frac{2hc^2}{\lambda^5}\frac{1}{\exp(hc/\lambda\kb T)-1}.
\end{align}

\subsection*{推导要点(光子占据数)}
在 TE 中,频率 \(\nu\) 的电磁模每个自由度平均占据数为
\begin{equation}
\bar n_\nu = \frac{1}{\exp(h\nu/\kb T)-1}.
\end{equation}
单位体积单位频率的模密度为 \(8\pi\nu^2/c^3\),每个光子能量 \(h\nu\),得能量密度
\begin{equation}
u_\nu = \frac{8\pi h\nu^3}{c^3}\frac{1}{\exp(h\nu/\kb T)-1}.
\end{equation}
各向同性辐射场中 \(u_\nu = 4\pi B_\nu/c\),从而得到 \(B_\nu\)。

\subsection*{常用极限与积分}
\begin{itemize}
  \item \textbf{Rayleigh--Jeans}(\(h\nu\ll \kb T\)):
  \(B_\nu \approx 2\nu^2\kb T/c^2\)。
  \item \textbf{Wien}(\(h\nu\gg \kb T\)):
  \(B_\nu \approx (2h\nu^3/c^2)\exp(-h\nu/\kb T)\)。
  \item \textbf{Stefan--Boltzmann}:\(\int_0^\infty B_\nu\,\dd\nu = \sigma T^4/\pi\)。
  \item \textbf{Wien 位移}:\(\lambda_{\rm peak}T \approx 2.9\times 10^{-3}\,\si{m\,K}\)(按 \(B_\lambda\) 峰值)。
\end{itemize}

\subsection*{天文学意义}
普朗克函数是 LTE 下源函数 \(S_\nu\) 的核心(见第 \ref{sec:lte} 节),也决定了连续谱形状、颜色温度以及临边昏暗的深层原因。

\section{吸收系数(Absorption coefficient)}
\subsection*{定义}
沿射线方向 \(s\) 的辐射转移:
\begin{equation}
\frac{\dd I_\nu}{\dd s} = -\alpha_\nu I_\nu + j_\nu,
\label{eq:rte_s}
\end{equation}
其中 \(\alpha_\nu\) 是\textbf{吸收系数}(单位 \(\si{cm^{-1}}\)),表示单位路径长度上强度的衰减率。

\subsection*{与不透明度、密度的关系}
天体物理中常用质量不透明度(质量吸收/消光系数)
\begin{equation}
\kappa_\nu \equiv \frac{\alpha_\nu}{\rho}\qquad [\si{cm^2\,g^{-1}}],
\end{equation}
并把总消光拆为
\begin{equation}
\alpha_\nu = \alpha_{\nu}^{\rm abs} + \alpha_{\nu}^{\rm sca}
\quad\Longleftrightarrow\quad
\kappa_\nu = \kappa_{\nu}^{\rm abs} + \kappa_{\nu}^{\rm sca}.
\end{equation}
有时也定义\textbf{单次相互作用后被“热化”的概率}(热化参数)
\begin{equation}
\epsilon_\nu \equiv \frac{\alpha_\nu^{\rm abs}}{\alpha_\nu^{\rm abs}+\alpha_\nu^{\rm sca}},
\end{equation}
它控制了散射主导时源函数对 \(J_\nu\) 的依赖(见下一节)。

\subsection*{微观表达}
\(\alpha_\nu\) 可写为“单位体积中吸收体数密度 \(\times\) 单个粒子的有效截面”之和:
\begin{equation}
\alpha_\nu = \sum_k n_k \sigma_{\nu,k},
\end{equation}
包含束缚--束缚(线吸收)、束缚--自由(光致电离/连续吸收)、自由--自由(制动辐射吸收)与散射(常以“散射吸收系数” \(\alpha_{\nu,{\rm sca}}\) 计入总消光)。

\subsection*{束缚--束缚(谱线)吸收的常用写法}
对跃迁 \(l\to u\),用振子强度 \(f_{lu}\) 表示的线吸收系数常写为
\begin{equation}
\alpha_\nu^{\rm line}
=
\frac{\pi e^2}{m_e c}\, f_{lu}\,
n_l\left(1-\frac{g_l n_u}{g_u n_l}\right)\phi(\nu),
\end{equation}
括号项是\textbf{受激发射修正}(当布居反转时可为负,对应受激放大/激射条件)。

\subsection*{天文学意义}
\(\alpha_\nu\) 决定在给定频率处“看见”大气的哪一层(光学深度 \(\tau_\nu\sim 1\)),从而决定连续谱与谱线形成层与对温度/压强/速度场的敏感性。

\section{发射系数(Emission coefficient)}
\subsection*{定义}
式 \eqref{eq:rte_s} 中 \(j_\nu\) 为\textbf{发射系数}(单位 \(\si{erg\,s^{-1}\,cm^{-3}\,Hz^{-1}\,sr^{-1}}\)),表示单位体积单位立体角单位频率的自发辐射贡献。

\subsection*{源函数}
定义源函数
\begin{equation}
S_\nu \equiv \frac{j_\nu}{\alpha_\nu}.
\label{eq:source}
\end{equation}
则辐射转移方程可写为
\begin{equation}
\frac{\dd I_\nu}{\dd s} = -\alpha_\nu \left(I_\nu - S_\nu\right).
\end{equation}
在 LTE 下(见第 \ref{sec:lte} 节)有 \(S_\nu=B_\nu(T)\)(Kirchhoff 定律)。

\subsection*{包含散射时的源函数(常用近似)}
若把“真正吸收”(会与热库交换能量)与“弹性散射”(只改传播方向/频率很小变化)区分开,在各向同性、相干散射近似下,源函数可写成
\begin{equation}
S_\nu = (1-\epsilon_\nu)J_\nu + \epsilon_\nu B_\nu(T),
\end{equation}
其中 \(J_\nu\) 是平均强度,\(\epsilon_\nu\) 为上一节定义的热化参数。
这条式子把“辐射场的非局域性”(通过 \(J_\nu\))与“局域热化”(通过 \(B_\nu\))明确分开,是理解散射主导大气(热星电子散射、强线散射)中谱线源函数行为的核心。

\subsection*{天文学意义}
源函数把“微观辐射产生机制”浓缩成一个量:谱线形成(尤其 NLTE)本质上是理解 \(S_\nu\) 如何偏离 \(B_\nu\)。

\section{光致电离(Photoionization)}
\subsection*{过程}
光子能量 \(h\nu\) 大于某能级电离阈值 \(\chi\) 时:
\[
\mathrm{X}^{i} + h\nu \rightarrow \mathrm{X}^{i+1} + e^-.
\]
它属于束缚--自由(bound-free)过程,是许多波段连续不透明度的重要来源。

\subsection*{光致电离率(把辐射场写进来)}
对某一初态 \(k\)(阈频 \(\nu_0\)),光致电离速率(单位 \(\si{s^{-1}}\))常写为
\begin{equation}
R_{k\to c} = \int_{\nu_0}^{\infty}\frac{4\pi J_\nu}{h\nu}\,\sigma_{\nu,k}^{\rm bf}\,\dd\nu.
\end{equation}
这显示:即使局部温度不高,只要紫外辐射场 \(J_\nu\) 强,仍可能发生\textbf{过电离}(over-ionization),是许多 NLTE 效应的典型来源。

\subsection*{截面与阈值行为}
光致电离截面 \(\sigma_{\nu}^{\rm bf}\) 在阈值 \(\nu_0=\chi/h\) 以上通常随频率下降(氢样近似 \(\sigma\propto \nu^{-3}\) 量级),并可能存在共振结构。

\subsection*{与反过程的关系}
反过程为\textbf{辐射复合}(radiative recombination),在 TE/LTE 下二者满足详细平衡,从而保证 Kirchhoff 定律与 Saha 关系一致。

\subsection*{天文学意义}
\begin{itemize}
  \item \textbf{连续谱形成}:例如热星紫外区金属光致电离、冷星可见光区 \(\mathrm{H^-}\) 的光致剥离(见第 \ref{sec:cont_sources} 节)。
  \item \textbf{电离平衡/NLTE}:在外层低密度区,光致电离常主导布居,导致 Saha 偏离与谱线强度变化。
\end{itemize}

\section{光致激发(Photoexcitation)}
\subsection*{过程}
光子诱导从低能级 \(l\) 到高能级 \(u\) 的跃迁(束缚--束缚):
\[
\mathrm{X}(l)+h\nu \rightarrow \mathrm{X}(u),
\qquad h\nu = E_u - E_l.
\]
对应吸收线的基本机制之一。

\subsection*{爱因斯坦系数与吸收}
跃迁概率可用爱因斯坦 \(B_{lu}\) 表示;线吸收系数常写为
\begin{equation}
\alpha_\nu^{\rm line} = \frac{h\nu}{4\pi}\left(n_l B_{lu} - n_u B_{ul}\right)\phi(\nu),
\label{eq:line_alpha}
\end{equation}
其中 \(\phi(\nu)\) 是谱线轮廓函数(见第 \ref{sec:line_profile} 节),括号中的差体现\textbf{受激发射}对净吸收的抵消。

\subsection*{天文学意义}
光致激发把辐射场与能级布居直接耦合,是 NLTE 的根源之一:辐射场由“远处的层”决定,布居因此具有非局域性。

\section{自由--自由吸收(Free-free / Bremsstrahlung absorption)}
\subsection*{过程}
自由电子在离子库仑场中散射时吸收/发射光子:
\[
e^- + \mathrm{X}^{i+1} + h\nu \rightleftharpoons e^- + \mathrm{X}^{i+1}.
\]
它没有阈值(连续),在高温、长波(红外/射电)往往重要。

\subsection*{基本依赖关系(量级)}
经典近似下自由--自由不透明度随
\begin{equation}
\alpha_\nu^{\rm ff}\propto n_e n_i\, T^{-1/2}\,\nu^{-3}\left(1-\exp\!\left[-\frac{h\nu}{\kb T}\right]\right)\, g_{\rm ff},
\end{equation}
其中 \(g_{\rm ff}\) 是 Gaunt 因子(量子修正,量级 \(\sim 1\))。

\subsection*{对应的自由--自由发射(并与 Kirchhoff 对照)}
自由--自由发射系数满足(量级)
\begin{equation}
j_\nu^{\rm ff}\propto n_e n_i\, T^{-1/2}\,\exp\!\left(-\frac{h\nu}{\kb T}\right)\, g_{\rm ff}.
\end{equation}
在 LTE 下可验证 \(j_\nu^{\rm ff}=\alpha_\nu^{\rm ff}B_\nu(T)\),这正是 Kirchhoff 定律在具体微观过程上的体现。

\subsection*{天文学意义}
\begin{itemize}
  \item \textbf{H\,II 区/恒星风射电连续谱}:自由--自由发射与吸收共同决定谱指数。
  \item \textbf{热星大气/色球}:高温电子使 ff 过程成为重要连续源。
\end{itemize}

\section{光学深度(Optical depth)}
\subsection*{定义}
定义沿路径的光学深度
\begin{equation}
\dd \tau_\nu \equiv \alpha_\nu \dd s,
\qquad
\tau_\nu(s) = \int_{s}^{\infty}\alpha_\nu \dd s'.
\end{equation}
用柱质量 \(m\)(见式 \eqref{eq:hse_m})也常写成
\begin{equation}
\dd \tau_\nu = \kappa_\nu \dd m,
\end{equation}
强调“形成深度”与质量柱密度/不透明度之间的联系。
则辐射转移方程 \eqref{eq:rte_s} 可写为
\begin{equation}
\frac{\dd I_\nu}{\dd \tau_\nu} = I_\nu - S_\nu
\end{equation}
(符号取决于 \(\tau\) 的积分方向;上式对应 \(\tau\) 向外递减的常用约定)。

\subsection*{物理意义:\(\tau_\nu\sim 1\) 的“可见层”}
\(\tau_\nu\) 衡量“光子被消光的累积概率”。经验上,某频率的出射辐射主要来自 \(\tau_\nu\approx 1\) 附近的层(更深处被强烈吸收、散射,多次相互作用后才逃逸)。

\subsection*{Eddington--Barbier 近似(非常常用)}
若源函数 \(S_\nu(\tau_\nu)\) 随深度变化较平滑,则形式解可近似为
\begin{equation}
I_\nu(0,\mu)\approx S_\nu(\tau_\nu=\mu),
\end{equation}
即“沿方向 \(\mu\) 的出射强度大致等于光学深度 \(\tau_\nu=\mu\) 处的源函数”。这条近似把\textbf{临边昏暗}与\textbf{谱线形成层}的直觉联系变成了可计算关系。

\subsection*{天文学意义}
\(\tau_\nu\) 提供了从几何深度到“辐射形成深度”的映射,使不同频率/谱线的形成层可比较(连续 vs 线芯 vs 线翼)。

\section{光子的自由程(Mean free path)}
\subsection*{定义与关系}
光子平均自由程
\begin{equation}
\ell_\nu \equiv \frac{1}{\alpha_\nu}.
\end{equation}
因此 \(\Delta\tau_\nu \sim 1\) 对应的几何尺度约为 \(\ell_\nu\)。

\subsection*{物理意义}
\(\ell_\nu\) 表示光子在发生一次“相互作用事件”(吸收/散射)前可自由传播的典型距离。外层密度降低导致 \(\alpha_\nu\) 下降、\(\ell_\nu\) 上升,辐射场更具非局域性(NLTE 更显著)。

\subsection*{随机游走与扩散近似的尺度}
在纯散射(或强散射)介质中,光子逃逸近似为随机游走:若从 \(\tau\) 深处出发,典型相互作用次数 \(N\sim \tau^2\),总路径长度 \(\sim N\ell\)。这解释了为何深层辐射可用扩散近似描述、而外层辐射则需要完整转移方程。

\section{不透明度的微观物理过程(Opacity microphysics)}
\subsection*{总消光与拆分}
通常定义总消光系数(不透明度对应的线性形式)
\begin{equation}
\alpha_\nu = \alpha_\nu^{\rm abs} + \alpha_\nu^{\rm sca}.
\end{equation}
若用质量不透明度 \(\kappa_\nu\)(单位 \(\si{cm^2\,g^{-1}}\)),则 \(\alpha_\nu = \rho\kappa_\nu\)。

\subsection*{主要过程清单}
\begin{itemize}
  \item \textbf{束缚--束缚(bound-bound)}:谱线吸收/发射(光致激发与受激发射),决定线谱与线 blanketing。
  \item \textbf{束缚--自由(bound-free)}:光致电离与辐射复合,决定连续边(如 Balmer jump)与紫外连续不透明度。
  \item \textbf{自由--自由(free-free)}:制动辐射吸收/发射,常在红外/射电重要。
  \item \textbf{散射(scattering)}:
    \begin{itemize}
      \item \textbf{汤姆孙散射}(自由电子):近似频率无关,\(\sigma_T=6.65\times10^{-25}\,\si{cm^2}\)。
      \item \textbf{瑞利散射}(束缚电子):\(\sigma\propto \lambda^{-4}\)。
      \item \textbf{线散射}:在谱线频率附近的共振散射。
    \end{itemize}
  \item \textbf{分子与尘埃}:低温大气中分子带与尘埃连续不透明度显著(晚型星/褐矮)。
\end{itemize}

\subsection*{天文学意义}
不透明度的频率依赖决定温度结构(反演/增温)、能量在谱上的再分配(blanketing)以及不同波段对元素丰度、湍动、重力的敏感性。

\section{恒星大气的基本方程组}
\subsection*{(1)辐射转移方程}
沿方向余弦 \(\mu=\cos\theta\),用平面平行近似的光学深度坐标 \(\tau_\nu\):
\begin{equation}
\mu\frac{\dd I_\nu(\tau_\nu,\mu)}{\dd \tau_\nu} = I_\nu(\tau_\nu,\mu)-S_\nu(\tau_\nu,\mu).
\label{eq:rte_plane}
\end{equation}

\subsection*{辐射场的角矩:\(J_\nu,H_\nu,K_\nu\)}
定义角矩(平面平行)
\begin{align}
J_\nu &\equiv \frac{1}{2}\int_{-1}^{1} I_\nu(\mu)\,\dd\mu,\\
H_\nu &\equiv \frac{1}{2}\int_{-1}^{1} \mu\, I_\nu(\mu)\,\dd\mu,\\
K_\nu &\equiv \frac{1}{2}\int_{-1}^{1} \mu^2\, I_\nu(\mu)\,\dd\mu.
\end{align}
它们与常用物理量的关系是
\begin{equation}
F_\nu = 4\pi H_\nu,\qquad
u_\nu = \frac{4\pi}{c}J_\nu,\qquad
P_{\nu,{\rm rad}} = \frac{4\pi}{c}K_\nu.
\end{equation}
为了从矩方程闭合,需给出 Eddington 因子 \(f_\nu \equiv K_\nu/J_\nu\)(Eddington 近似取 \(f_\nu=1/3\))。

\subsection*{(2)流体静力平衡(或动量方程)}
静态、平面平行时
\begin{equation}
\frac{\dd P}{\dd z} = -\rho g,
\end{equation}
或用柱质量 \(m\equiv \int_z^\infty \rho\,\dd z\)(\(\dd m=-\rho \dd z\))写为
\begin{equation}
\frac{\dd P}{\dd m} = g.
\label{eq:hse_m}
\end{equation}
必要时加入辐射压梯度或动压项(强辐射场、恒星风)。

\subsection*{(3)能量守恒/辐射平衡}
纯辐射平衡时,总通量(频率积分)守恒:
\begin{equation}
F \equiv \int_0^\infty F_\nu\,\dd\nu = \sigma \Teff^4 = \text{const}.
\label{eq:flux_const}
\end{equation}
若有对流,则总通量 \(F=F_{\rm rad}+F_{\rm conv}\)。

\subsection*{辐射平衡的等价“积分形式”}
由 \(\dd H_\nu/\dd\tau_\nu = J_\nu-S_\nu\)(矩方程)可得辐射平衡条件等价于
\begin{equation}
\int_0^\infty \alpha_\nu^{\rm abs}\,\left(J_\nu-S_\nu\right)\,\dd\nu = 0.
\end{equation}
在 LTE 下 \(S_\nu=B_\nu(T)\),上式表示“单位体积的吸收能量率等于发射能量率”(净加热为零)。

\subsection*{(4)物态方程与电离/激发}
\begin{equation}
P = \frac{\rho}{\mu m_{\rm H}}\kb T + P_{\rm rad} + \cdots,
\end{equation}
并结合 Saha(第 \ref{eq:saha} 式)、玻尔兹曼(第 \ref{eq:boltzmann} 式)或更一般的统计平衡方程(NLTE)。

\subsection*{统计平衡(NLTE 的核心方程)}
若不强制 Saha/玻尔兹曼,能级布居由统计平衡决定:
\begin{equation}
\sum_{j\neq i} n_j P_{ji} = n_i \sum_{j\neq i} P_{ij},
\end{equation}
其中 \(P_{ij}=R_{ij}+C_{ij}\) 是从 \(i\to j\) 的总跃迁率(辐射 \(R\) 与碰撞 \(C\))。这套方程与辐射转移方程 \eqref{eq:rte_plane} 强耦合,构成 NLTE 大气模型的计算难点。

\subsection*{(5)化学组成与电中性}
给定元素丰度,满足电中性 \(\sum_i Z_i n_i = n_e\),并与分子平衡(低温)耦合。

\subsection*{边界条件(常用)}
\begin{itemize}
  \item \textbf{外边界}:入射辐射 \(I_\nu(\tau_\nu=0,\mu<0)\approx 0\)。
  \item \textbf{深层}:扩散近似 \(I_\nu \approx B_\nu + \mathcal{O}(\tau_\nu^{-1})\)。
\end{itemize}

\section{灰大气模型(Grey atmosphere model)}
\subsection*{基本假设}
\begin{itemize}
  \item \textbf{灰不透明度}:质量不透明度 \(\kappa_\nu\) 取为常数 \(\kappa\)(与频率无关)。
  \item \textbf{平面平行、静态、辐射平衡}:\(F=\sigma \Teff^4\) 常数。
  \item \textbf{LTE}:源函数 \(S_\nu=B_\nu(T)\)。
  \item \textbf{闭合近似}:常用 Eddington 近似 \(K=\tfrac{1}{3}J\)。
\end{itemize}
灰大气是理解温度结构、临边昏暗的最小模型,也是更复杂 NLTE/非灰模型的参照。

\subsection*{Rosseland 平均与“灰深度”}
在深层扩散近似下,能量输运更接近“按温度梯度加权的频率平均”。因此常定义 Rosseland 平均不透明度:
\begin{equation}
\frac{1}{\kappa_R}
=
\frac{\displaystyle \int_0^\infty \frac{1}{\kappa_\nu}\,\frac{\partial B_\nu}{\partial T}\,\dd\nu}
{\displaystyle \int_0^\infty \frac{\partial B_\nu}{\partial T}\,\dd\nu}.
\end{equation}
并以 \(\dd\tau \equiv \kappa_R \dd m\) 定义 Rosseland 光学深度 \(\tau\)。
灰大气模型把 \(\kappa_\nu\) 近似为常数,相当于 \(\kappa_R\) 也为常数,从而频率积分的方程可以用单一 \(\tau\) 描述。

\subsection*{频率积分后的矩方程}
对 \eqref{eq:rte_plane} 取角矩并频率积分,得到(略去推导细节):
\begin{align}
\frac{\dd H}{\dd \tau} &= J - S,\\
\frac{\dd K}{\dd \tau} &= H,
\end{align}
其中 \(\tau\) 为 Rosseland 光学深度(灰模型下与频率无关),并且在辐射平衡且 LTE 下 \(S=J\),因此
\begin{equation}
\frac{\dd H}{\dd \tau}=0 \Rightarrow H=\text{const}.
\end{equation}
通量 \(F=4\pi H\) 常数。

\section{灰大气模型的温度分布}
\subsection*{Eddington 近似推导}
在 LTE 与辐射平衡下,频率积分源函数 \(S=\int S_\nu\,\dd\nu \equiv B=\sigma T^4/\pi\),并有 \(J=S=B\)。
由矩方程 \(\dd K/\dd\tau = H\) 积分得
\begin{equation}
K(\tau) = H\tau + C.
\end{equation}
Eddington 近似给出 \(K=\tfrac{1}{3}J=\tfrac{1}{3}B\),从而
\begin{equation}
\frac{1}{3}B(\tau) = H\tau + C.
\end{equation}
将 \(H=F/(4\pi)=\sigma \Teff^4/(4\pi)\)、\(B=\sigma T^4/\pi\) 代入:
\begin{equation}
\frac{1}{3}\frac{\sigma T^4}{\pi} = \frac{\sigma \Teff^4}{4\pi}\tau + C.
\end{equation}
两边乘以 \(\pi/\sigma\):
\begin{equation}
\frac{1}{3}T^4 = \frac{1}{4}\Teff^4 \tau + C',
\end{equation}
即
\begin{equation}
T^4(\tau)=\frac{3}{4}\Teff^4\left(\tau+q\right),
\label{eq:grey_T}
\end{equation}
其中常数 \(q\) 由边界条件确定。

\subsection*{表面边界条件与 \(q=2/3\)}
采用 Eddington 边界条件(半空间、外边界无入射),可得 \(J(0)=2H\),即
\[
\frac{1}{3}J(0)=\frac{2}{3}H.
\]
又 \(K(0)=\tfrac{1}{3}J(0)\),而 \(K(0)=C\),故 \(C=\tfrac{2}{3}H\),对应 \(q=2/3\)。
因此灰大气的经典温度结构为
\begin{equation}
T^4(\tau)=\frac{3}{4}\Teff^4\left(\tau+\frac{2}{3}\right).
\end{equation}

\subsection*{天文学意义}
\begin{itemize}
  \item \textbf{形成深度}:连续谱的出射辐射大致采样 \(\tau\sim 1\) 附近,从而 \(T(\tau\!\sim\!1)\approx 1.06\,\Teff\)。
  \item \textbf{临边昏暗}:由于不同 \(\mu\) 看到不同 \(\tau\) 层,温度随 \(\tau\) 递增导致中心更亮(见下一节)。
  \item \textbf{局限}:真实大气是非灰的,\(\kappa_\nu\) 变化会引起温度反演、线 blanketing 等。
\end{itemize}

\section{太阳临边昏暗现象(Limb darkening)}
\subsection*{现象与直观解释}
太阳圆盘中心比边缘更亮:边缘视线倾斜(小 \(\mu\))时,光程更长、更容易在较高更冷的层达到 \(\tau_\nu\sim 1\),因此出射强度更小。

\subsection*{从形式解出发}
平面平行下的出射强度(\(\tau=0\) 处,\(\mu>0\))满足形式解:
\begin{equation}
I_\nu(0,\mu)=\frac{1}{\mu}\int_0^\infty S_\nu(\tau_\nu)\,\exp\!\left(-\frac{\tau_\nu}{\mu}\right)\,\dd\tau_\nu.
\label{eq:formal_solution}
\end{equation}
若 \(S_\nu\) 随深度单调增大(通常由温度递增导致),则小 \(\mu\) 的指数权重 \(\exp(-\tau/\mu)\) 更偏向浅层,从而 \(I_\nu(0,\mu)\) 更小。

\subsection*{Eddington--Barbier 直觉:为何“看到的是 \(\tau_\nu\approx\mu\) 的层”}
把源函数在 \(\tau_\nu=\mu\) 附近做一阶展开并代入 \eqref{eq:formal_solution},可得近似
\begin{equation}
I_\nu(0,\mu)\approx S_\nu(\tau_\nu=\mu),
\end{equation}
因此临边昏暗强弱直接由 \(S_\nu(\tau_\nu)\)(LTE 下近似 \(B_\nu[T(\tau_\nu)]\))的梯度控制。
由于 \(\kappa_\nu\) 随波长变化,\(\tau_\nu=\mu\) 对应的几何高度也随波长变化,因此\textbf{临边昏暗具有显著的波长依赖}(通常蓝光更强)。

\subsection*{灰大气的一个解析例子(频率积分)}
在灰大气中 \(S=B\propto T^4\propto (\tau+2/3)\),因此可近似取
\begin{equation}
S(\tau)=S_0 + S_1 \tau,
\end{equation}
代入 \eqref{eq:formal_solution} 并积分可得线性临边昏暗形式(频率积分后):
\begin{equation}
I(0,\mu) \propto \left(\mu+\frac{2}{3}\right),
\qquad
\Rightarrow\quad
\frac{I(0,\mu)}{I(0,1)} \approx \frac{\mu+2/3}{1+2/3}.
\end{equation}
真实太阳的临边昏暗系数随波长变化(蓝光更强),需要非灰不透明度与真实温度梯度描述。

\subsection*{天文学意义}
临边昏暗影响凌星光变曲线、恒星半径与行星参数反演;也影响干涉测量的可见度函数与“角直径”定义。

\section{局部热动平衡大气模型(LTE)}
\label{sec:lte}
\subsection*{定义}
LTE 假设:\textbf{物质}在每一点的微观布居由局部温度 \(T(\bm r)\) 决定(Saha+玻尔兹曼成立),并且满足 Kirchhoff 定律:
\begin{equation}
S_\nu = B_\nu(T).
\end{equation}
但 \textbf{辐射场}不必是黑体:\(J_\nu\neq B_\nu\) 可以成立。

\subsection*{成立条件(经验)}
\begin{itemize}
  \item 碰撞过程足够频繁,使能级布居主要由碰撞热化决定(高密度、较深层)。
  \item 光子平均自由程较短,辐射场更“局域”。
\end{itemize}

\subsection*{何时失效(NLTE)}
在外层低密度、强辐射场(紫外)或稀薄等离子体中,辐射跃迁率可超过碰撞率,导致过电离、源函数偏离普朗克函数等 NLTE 效应,从而显著改变谱线强度与轮廓。

\section{氢原子模型}
\subsection*{氢样能级与简并度}
在忽略精细结构与外场的最简模型中:
\begin{equation}
E_n = -\frac{13.6\,\si{eV}}{n^2},
\qquad n=1,2,\dots
\end{equation}
主量子数 \(n\) 的简并度(含自旋)为 \(g_n=2n^2\)。跃迁频率由能级差决定,形成 Lyman(到 \(n=1\))、Balmer(到 \(n=2\))等谱系。

\subsection*{选择定则与谱线}
电偶极跃迁满足 \(\Delta \ell=\pm 1\) 等选择定则。实际谱线还受精细结构、超精细结构、外场(Zeeman/Stark)与碰撞展宽影响。

\subsection*{在恒星大气中的角色}
\begin{itemize}
  \item \textbf{热星}:氢 Balmer 线翼对电子压强敏感,可用于测 \(g\)。
  \item \textbf{连续边}:Balmer 跳跃与 Lyman 连续强烈影响紫外/蓝端能量分布。
  \item \textbf{等离子体参量}:氢是最重要的电子供体与碰撞伙伴之一,影响 \(n_e\) 与不透明度。
\end{itemize}

\section{连续吸收的来源}
\label{sec:cont_sources}
\subsection*{主要连续不透明度机制}
\begin{itemize}
  \item \textbf{H$^-$ 连续}(晚型星可见光主导):
    \begin{itemize}
      \item \textbf{束缚--自由}:\(\mathrm{H^-}+h\nu\rightarrow \mathrm{H}+e^-\)(光致剥离)。
      \item \textbf{自由--自由}:\(\mathrm{H}+e^-+h\nu\rightarrow \mathrm{H}+e^-\)(H$^-$ ff)。
    \end{itemize}
  \item \textbf{H\,I/He\,I/He\,II 的束缚--自由}:形成 Lyman/Balmer/Paschen 等连续边,热星紫外非常重要。
  \item \textbf{自由--自由(电子--离子)}:红外/射电、热等离子体重要。
  \item \textbf{电子散射}:热星与高光度星中可成为重要灰散射源。
  \item \textbf{分子与尘埃}:低温大气(M 型、褐矮)中分子连续/准连续与尘埃吸收显著。
\end{itemize}

\subsection*{为何 H$^-$ 如此关键(晚型星)}
虽然 \(\mathrm{H^-}\) 丰度不高,但在 \(\sim 0.3\text{--}1.6\,\si{\micro m}\) 具有较大的连续截面;并且晚型星光球层温度/电子密度条件使 \(\mathrm{H^-}\) 易形成,从而主导可见光连续谱,直接决定 \(\Teff\) 标定与颜色。

\section{吸收线的产生机制}
\subsection*{基本图景}
当某频率处存在束缚--束缚跃迁的强吸收(\(\alpha_\nu\) 在该频率显著增大)时,\(\tau_\nu\sim1\) 的形成层被“抬到”更高、更冷的层,导致该频率的出射强度(或通量)低于连续谱,形成吸收线。

\subsection*{线源函数与 NLTE}
在线频率处源函数一般为
\begin{equation}
S_\nu^{\rm line} = \frac{j_\nu^{\rm line}}{\alpha_\nu^{\rm line}},
\end{equation}
在 LTE 下 \(S_\nu^{\rm line}=B_\nu(T)\),吸收线强度主要由不透明度增强与温度梯度决定;在 NLTE 下 \(S_\nu^{\rm line}\) 可偏离 \(B_\nu\),出现线核填充、反转甚至发射线。

\subsection*{天文学意义}
谱线的深度与形状编码了元素丰度、温度结构、压力(展宽)、速度场(多普勒位移/不对称)与磁场(塞曼分裂)。

\section{谱线轮廓(Line profile)}
\label{sec:line_profile}
\subsection*{轮廓函数的定义与归一化}
把线吸收系数写为
\begin{equation}
\alpha_\nu^{\rm line} = \alpha_0\,\phi(\nu),
\qquad
\int_{-\infty}^{\infty}\phi(\nu)\,\dd\nu = 1,
\end{equation}
\(\phi(\nu)\) 描述吸收在频率上的分布。常用变量
\[
u\equiv \frac{\nu-\nu_0}{\Delta\nu_D}
\]
把频率偏移无量纲化。

\subsection*{热运动导致的多普勒(高斯)轮廓}
若吸收原子速度分量服从 Maxwell 分布,则频率偏移服从高斯:
\begin{equation}
\phi_D(\nu)=\frac{1}{\Delta\nu_D\sqrt{\pi}}\exp\!\left[-\left(\frac{\nu-\nu_0}{\Delta\nu_D}\right)^2\right],
\end{equation}
其中多普勒宽度
\begin{equation}
\Delta\nu_D=\frac{\nu_0}{c}\sqrt{\frac{2\kb T}{m}+\xi_{\rm turb}^2},
\end{equation}
可加入微湍动速度 \(\xi_{\rm turb}\)。

\subsection*{自然展宽与碰撞展宽(洛伦兹翼)}
有限寿命导致能级不确定性(自然展宽),碰撞导致相位扰动(压强展宽),二者常合并为阻尼常数 \(\Gamma\):
\begin{equation}
\phi_L(\nu)=\frac{1}{\pi}\frac{\gamma}{(\nu-\nu_0)^2+\gamma^2},
\qquad
\gamma\equiv \frac{\Gamma}{4\pi}.
\end{equation}

\section{等值宽度(Equivalent width)}
\subsection*{定义}
以连续谱通量 \(F_c\) 为基准,等值宽度定义为“与谱线吸收等面积的矩形宽度”:
\begin{equation}
W_\lambda \equiv \int \left(1-\frac{F_\lambda}{F_c}\right)\dd\lambda.
\label{eq:ew}
\end{equation}
若用强度也可定义 \(W_\lambda=\int (1-I_\lambda/I_c)\dd\lambda\)(需明确是通量还是强度)。

\subsection*{物理意义}
\(W_\lambda\) 将谱线“深度+宽度”的信息压缩成一个标量,对分辨率与展宽细节相对不那么敏感,因此常用于丰度分析与曲线增长(curve of growth)。

\subsection*{曲线增长的三段}
\begin{itemize}
  \item \textbf{弱线(线性段)}:\(\tau_0\ll1\),\(W\propto N\)(吸收原子柱密度/丰度)。
  \item \textbf{饱和段}:线芯饱和,增加 \(N\) 主要增加宽度,增长变慢(对微湍动敏感)。
  \item \textbf{阻尼翼段}:洛伦兹翼贡献主导,\(W\) 再次加速增长(对压强展宽敏感)。
\end{itemize}

\section{谱线致宽机制(Line broadening mechanisms)}
\subsection*{(1)热多普勒展宽}
由粒子热运动产生,尺度 \(\Delta\nu_D\propto \sqrt{T/m}\),轻元素线更宽。

\subsection*{(2)湍动展宽(微/宏湍动)}
\begin{itemize}
  \item \textbf{微湍动}:尺度小于光子平均自由程,进入局部吸收系数,等效增加 \(\Delta\nu_D\)。
  \item \textbf{宏湍动}:尺度大于形成区,主要通过对最终谱线的速度场卷积改变轮廓,不改等值宽度(理想化下)。
\end{itemize}

\subsection*{(3)自然展宽}
由激发态寿命 \(\tau\) 有限,\(\Gamma_{\rm nat}\sim 1/\tau\),形成洛伦兹线翼但通常线芯不主导(除极窄线/低温低压)。

\subsection*{(4)压强(碰撞)展宽}
与粒子密度相关,常见:
\begin{itemize}
  \item \textbf{van der Waals 展宽}:中性粒子碰撞(晚型星重要)。
  \item \textbf{Stark 展宽}:电子/离子电场扰动(热星、Balmer 线翼尤重要)。
\end{itemize}

\subsection*{(5)自转、脉动与大尺度速度场}
自转产生典型“旋转轮廓”并导致线展宽;径向速度场(风、脉动、对流)会造成线的不对称、位移与分裂(P Cygni 等)。

\subsection*{(6)仪器展宽}
由光谱仪线扩散函数(LSF)决定,观测谱线是天体真实谱线与 LSF 的卷积。

\section{Voigt 轮廓(Voigt profile)}
\subsection*{定义:高斯与洛伦兹的卷积}
实际谱线往往同时受多普勒与阻尼展宽,轮廓为卷积:
\begin{equation}
\phi_V(\nu)=\int_{-\infty}^{\infty}\phi_D(\nu')\,\phi_L(\nu-\nu')\,\dd\nu'.
\end{equation}
用无量纲变量
\[
u=\frac{\nu-\nu_0}{\Delta\nu_D},\qquad
a=\frac{\gamma}{\Delta\nu_D}
\]
可写为
\begin{equation}
\phi_V(\nu)=\frac{1}{\Delta\nu_D\sqrt{\pi}}\,H(a,u),
\end{equation}
其中 Voigt 函数
\begin{equation}
H(a,u)=\frac{a}{\pi}\int_{-\infty}^{\infty}\frac{\exp(-y^2)}{(u-y)^2+a^2}\,\dd y
=\Re\left[w(u+\mathrm{i}a)\right]
\end{equation}
(\(w\) 为 Faddeeva 函数)。

\subsection*{近似行为}
\begin{itemize}
  \item \textbf{线芯}:\(|u|\lesssim 1\) 时接近高斯;
  \item \textbf{线翼}:\(|u|\gg 1\) 时呈洛伦兹翼 \(H\sim a/(\sqrt{\pi}u^2)\)。
\end{itemize}

\subsection*{天文学意义}
Voigt 轮廓是大气谱线合成的标准输入:线芯诊断温度/微湍动与速度场,线翼诊断压强(重力)与碰撞物理;对高分辨率丰度分析与线形成深度判读至关重要。

\end{document}

